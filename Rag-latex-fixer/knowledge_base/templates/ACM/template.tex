%%
%% This is file `sample-sigplan.tex',
%% generated with the docstrip utility.
%%
%% The original source files were:
%%
%% samples.dtx  (with options: `sigplan')
%% 
%% IMPORTANT NOTICE:
%% 
%% For the copyright see the source file.
%% 
%% Any modified versions of this file must be renamed
%% with new filenames distinct from sample-sigplan.tex.
%% 
%% For distribution of the original source see the terms
%% for copying and modification in the file samples.dtx.
%% 
%% This generated file may be distributed as long as the
%% original source files, as listed above, are part of the
%% same distribution. (The sources need not necessarily be
%% in the same archive or directory.)
%%
%% Commands for TeXCount
%TC:macro \cite [option:text,text]
%TC:macro \citep [option:text,text]
%TC:macro \citet [option:text,text]
%TC:envir table 0 1
%TC:envir table* 0 1
%TC:envir tabular [ignore] word
%TC:envir displaymath 0 word
%TC:envir math 0 word
%TC:envir comment 0 0
%%
%%
%% The first command in your LaTeX source must be the \documentclass command.
\documentclass[sigplan,screen]{acmart}
\usepackage{bbm}
\usepackage{ulem}
\usepackage{amsmath}
% Define argmax
\DeclareMathOperator*{\argmax}{arg\,max}

% \newcommand{\hr}[1]{\textcolor{blue}{#1}}
\newcommand{\hr}[1]{\textcolor{black}{#1}}

%% NOTE that a single column version is required for 
%% submission and peer review. This can be done by changing
%% the \doucmentclass[...]{acmart} in this template to 
%% \documentclass[manuscript,screen,review]{acmart}
%% 
%% To ensure 100% compatibility, please check the white list of
%% approved LaTeX packages to be used with the Master Article Template at
%% https://www.acm.org/publications/taps/whitelist-of-latex-packages 
%% before creating your document. The white list page provides 
%% information on how to submit additional LaTeX packages for 
%% review and adoption.
%% Fonts used in the template cannot be substituted; margin 
%% adjustments are not allowed.
%%
%% \BibTeX command to typeset BibTeX logo in the docs
\AtBeginDocument{%
  \providecommand\BibTeX{{%
    \normalfont B\kern-0.5em{\scshape i\kern-0.25em b}\kern-0.8em\TeX}}}

%% Rights management information.  This information is sent to you
%% when you complete the rights form.  These commands have SAMPLE
%% values in them; it is your responsibility as an author to replace
%% the commands and values with those provided to you when you
%% complete the rights form.
\copyrightyear{2024}
\acmYear{2024}
\setcopyright{rightsretained}
\acmConference[DAC '24]{61st ACM/IEEE Design Automation Conference}{June 23--27, 2024}{San Francisco, CA, USA}
\acmBooktitle{61st ACM/IEEE Design Automation Conference (DAC '24), June 23--27, 2024, San Francisco, CA, USA}
\acmDOI{10.1145/3649329.3658473}
\acmISBN{979-8-4007-0601-1/24/06}

% https://doi.org/10.1145/3649329.3658473
% These commands are for a PROCEEDINGS abstract or paper.
\acmConference[DAC'24]{61st ACM/IEEE Design Automation Conference}{June 23--27, 2024}{San Francisco, CA}

%%
%% Submission ID.
%% Use this when submitting an article to a sponsored event. You'll
%% receive a unique submission ID from the organizers
%% of the event, and this ID should be used as the parameter to this command.
\acmSubmissionID{1122}

\widowpenalties 5 100 80 60 40 20
\raggedbottom

\begin{document}

%%
%% The "title" command has an optional parameter,
%% allowing the author to define a "short title" to be used in page headers.

\title{EDGE-LLM: Enabling Efficient Large Language Model Adaptation on Edge Devices via Layerwise Unified Compression and Adaptive Layer Tuning \& Voting}

% \newcommand{\titleorig}{EDGE-LLM: Enabling Efficient Large Language Model Adaptation on Edge Devices via Layerwise Unified Compression and Adaptive Layer Tuning \& Voting}


\author{\small Zhongzhi Yu$^1$, Zheng Wang$^1$, Yuhan Li$^1$, Haoran You$^1$, Ruijie Gao$^1$,  Xiaoya Zhou$^3$, Sreenidhi Reedy Bommu$^1$, Yang (Katie) Zhao$^2$, Yingyan (Celine) Lin$^1$}
\affiliation{%
  \institution{$^1$\textit{Georgia Institute of Technology}, $^2$\textit{University of Minnesota, Twin Cities},
  $^3$\textit{University of California, Santa Barbara}}
  \city{\{zyu401, zwang3478, yli3326, hyou37, eiclab.gatech, sbommu3, celine.lin\}@gatech.edu,\\ yangzhao@umn.edu, xiaoyazhou@umail.ucsb.edu}
  \country{}
}

%%
%% The "author" command and its associated commands are used to define
%% the authors and their affiliations.
%% Of note is the shared affiliation of the first two authors, and the
%% "authornote" and "authornotemark" commands
%% used to denote shared contribution to the research.


%%
%% By default, the full list of authors will be used in the page
%% headers. Often, this list is too long, and will overlap
%% other information printed in the page headers. This command allows
%% the author to define a more concise list
%% of authors' names for this purpose.
\renewcommand{\shortauthors}{Zhongzhi Yu, et al.}
\renewcommand{\shorttitle}{Edge-LLM}

%%
%% The abstract is a short summary of the work to be presented in the
%% article.

% \settopmatter{printacmref=false} % Removes citation information below abstract
% \renewcommand\footnotetextcopyrightpermission[1]{} % removes footnote with conference information in first column
% \pagestyle{plain} % removes running headers
% \vspace{-1em}
\input{Sections/0-Abstract}



%% A "teaser" image appears between the author and affiliation
%% information and the body of the document, and typically spans the
%% page.
% \begin{teaserfigure}
%   \includegraphics[width=\textwidth]{sampleteaser}
%   \caption{Seattle Mariners at Spring Training, 2010.}
%   \Description{Enjoying the baseball game from the third-base
%   seats. Ichiro Suzuki preparing to bat.}
%   \label{fig:teaser}
% \end{teaserfigure}

% \received{20 February 2007}
% \received[revised]{12 March 2009}
% \received[accepted]{5 June 2009}

%%
%% This command processes the author and affiliation and title
%% information and builds the first part of the formatted document.
\maketitle

\input{Sections/1-Introduction}

\input{Sections/2-Preliminary}

\input{Sections/3-Algorithm}

\input{Sections/4-Hardware}

\input{Sections/5-Evaluation}

\input{Sections/6-Conclusion}

%%
%% The next two lines define the bibliography style to be used, and
%% the bibliography file.
\vspace{-0.5em}
\bibliographystyle{ACM-Reference-Format}
\bibliography{sample-base}

%%

\end{document}
\endinput
%%
%% End of file `sample-sigplan.tex'.
%%
%% This is file `sample-sigplan.tex',
%% generated with the docstrip utility.
%%
%% The original source files were:
%%
%% samples.dtx  (with options: `sigplan')
%% 
%% IMPORTANT NOTICE:
%% 
%% For the copyright see the source file.
%% 
%% Any modified versions of this file must be renamed
%% with new filenames distinct from sample-sigplan.tex.
%% 
%% For distribution of the original source see the terms
%% for copying and modification in the file samples.dtx.
%% 
%% This generated file may be distributed as long as the
%% original source files, as listed above, are part of the
%% same distribution. (The sources need not necessarily be
%% in the same archive or directory.)
%%
%% Commands for TeXCount
%TC:macro \cite [option:text,text]
%TC:macro \citep [option:text,text]
%TC:macro \citet [option:text,text]
%TC:envir table 0 1
%TC:envir table* 0 1
%TC:envir tabular [ignore] word
%TC:envir displaymath 0 word
%TC:envir math 0 word
%TC:envir comment 0 0
%%
%%
%% The first command in your LaTeX source must be the \documentclass command.
\documentclass[sigplan,screen]{acmart}
\usepackage{bbm}
\usepackage{ulem}

%% NOTE that a single column version is required for 
%% submission and peer review. This can be done by changing
%% the \doucmentclass[...]{acmart} in this template to 
%% \documentclass[manuscript,screen,review]{acmart}
%% 
%% To ensure 100% compatibility, please check the white list of
%% approved LaTeX packages to be used with the Master Article Template at
%% https://www.acm.org/publications/taps/whitelist-of-latex-packages 
%% before creating your document. The white list page provides 
%% information on how to submit additional LaTeX packages for 
%% review and adoption.
%% Fonts used in the template cannot be substituted; margin 
%% adjustments are not allowed.
%%
%% \BibTeX command to typeset BibTeX logo in the docs
\AtBeginDocument{%
  \providecommand\BibTeX{{%
    \normalfont B\kern-0.5em{\scshape i\kern-0.25em b}\kern-0.8em\TeX}}}

%% Rights management information.  This information is sent to you
%% when you complete the rights form.  These commands have SAMPLE
%% values in them; it is your responsibility as an author to replace
%% the commands and values with those provided to you when you
%% complete the rights form.
\setcopyright{acmlicensed}
\copyrightyear{2024}
\acmYear{2024}
\acmDOI{XXXXXXX.XXXXXXX}

% These commands are for a PROCEEDINGS abstract or paper.
\acmConference[DAC'24]{61st ACM/IEEE Design Automation Conference}{June 23--27, 2024}{San Francisco, CA}

%%
%% Submission ID.
%% Use this when submitting an article to a sponsored event. You'll
%% receive a unique submission ID from the organizers
%% of the event, and this ID should be used as the parameter to this command.
\acmSubmissionID{1122}

\widowpenalties 5 100 80 60 40 20
\raggedbottom

\begin{document}

%%
%% The "title" command has an optional parameter,
%% allowing the author to define a "short title" to be used in page headers.

\title{\acmsmall EDGE-LLM: Enabling Efficient Large Language Model Adaptation on Edge Devices via Layerwise Unified Compression and Adaptive Layer Tuning \& Voting}

% \newcommand{\titleorig}{EDGE-LLM: Enabling Efficient Large Language Model Adaptation on Edge Devices via Layerwise Unified Compression and Adaptive Layer Tuning \& Voting}


\author{\small Zhongzhi Yu$^1$, Zheng Wang$^1$, Yuhan Li$^1$, Xiaoya Zhou$^3$, Haoran You$^1$,\\ Ruijie Gao$^1$, Sreenidhi Reedy Bommu$^1$, Yang (Katie) Zhao$^2$, Yingyan (Celine) Lin$^1$}
\affiliation{%
  \institution{$^1$\textit{Georgia Institute of Technology}, $^2$\textit{University of Minnesota, Twin Cities},
  $^3$\textit{University of California, Santa Barbara}}
  \city{\{zyu401, zwang2478, yli3326, hyou37, eiclab.gatech, sbommu3, celine.lin\}@gatech.edu,\\ yangzhao@umn.edu, xiaoyazhou@umail.ucsb.edu}
  \country{}
}

%%
%% The "author" command and its associated commands are used to define
%% the authors and their affiliations.
%% Of note is the shared affiliation of the first two authors, and the
%% "authornote" and "authornotemark" commands
%% used to denote shared contribution to the research.


%%
%% By default, the full list of authors will be used in the page
%% headers. Often, this list is too long, and will overlap
%% other information printed in the page headers. This command allows
%% the author to define a more concise list
%% of authors' names for this purpose.
\renewcommand{\shortauthors}{Zhongzhi Yu, et al.}
\renewcommand{\shorttitle}{Edge-LLM}

%%
%% The abstract is a short summary of the work to be presented in the
%% article.

% \settopmatter{printacmref=false} % Removes citation information below abstract
% \renewcommand\footnotetextcopyrightpermission[1]{} % removes footnote with conference information in first column
% \pagestyle{plain} % removes running headers
% \vspace{-1em}
\input{Sections/0-Abstract}



%% A "teaser" image appears between the author and affiliation
%% information and the body of the document, and typically spans the
%% page.
% \begin{teaserfigure}
%   \includegraphics[width=\textwidth]{sampleteaser}
%   \caption{Seattle Mariners at Spring Training, 2010.}
%   \Description{Enjoying the baseball game from the third-base
%   seats. Ichiro Suzuki preparing to bat.}
%   \label{fig:teaser}
% \end{teaserfigure}

% \received{20 February 2007}
% \received[revised]{12 March 2009}
% \received[accepted]{5 June 2009}

%%
%% This command processes the author and affiliation and title
%% information and builds the first part of the formatted document.
\maketitle

\input{Sections/1-Introduction}

\input{Sections/2-Preliminary}

\input{Sections/3-Algorithm}

\input{Sections/4-Hardware}

\input{Sections/5-Evaluation}

\input{Sections/6-Conclusion}

%%
%% The next two lines define the bibliography style to be used, and
%% the bibliography file.
\vspace{-1em}
\bibliographystyle{ACM-Reference-Format}
\bibliography{sample-base}

%%

\end{document}
\endinput
%%
%% End of file `sample-sigplan.tex'.
%%
%% This is file `sample-sigplan.tex',
%% generated with the docstrip utility.
%%
%% The original source files were:
%%
%% samples.dtx  (with options: `sigplan')
%% 
%% IMPORTANT NOTICE:
%% 
%% For the copyright see the source file.
%% 
%% Any modified versions of this file must be renamed
%% with new filenames distinct from sample-sigplan.tex.
%% 
%% For distribution of the original source see the terms
%% for copying and modification in the file samples.dtx.
%% 
%% This generated file may be distributed as long as the
%% original source files, as listed above, are part of the
%% same distribution. (The sources need not necessarily be
%% in the same archive or directory.)
%%
%% Commands for TeXCount
%TC:macro \cite [option:text,text]
%TC:macro \citep [option:text,text]
%TC:macro \citet [option:text,text]
%TC:envir table 0 1
%TC:envir table* 0 1
%TC:envir tabular [ignore] word
%TC:envir displaymath 0 word
%TC:envir math 0 word
%TC:envir comment 0 0
%%
%%
%% The first command in your LaTeX source must be the \documentclass command.
\documentclass[sigplan,screen]{acmart}
\usepackage{bbm}
\usepackage{ulem}

%% NOTE that a single column version is required for 
%% submission and peer review. This can be done by changing
%% the \doucmentclass[...]{acmart} in this template to 
%% \documentclass[manuscript,screen,review]{acmart}
%% 
%% To ensure 100% compatibility, please check the white list of
%% approved LaTeX packages to be used with the Master Article Template at
%% https://www.acm.org/publications/taps/whitelist-of-latex-packages 
%% before creating your document. The white list page provides 
%% information on how to submit additional LaTeX packages for 
%% review and adoption.
%% Fonts used in the template cannot be substituted; margin 
%% adjustments are not allowed.
%%
%% \BibTeX command to typeset BibTeX logo in the docs
\AtBeginDocument{%
  \providecommand\BibTeX{{%
    \normalfont B\kern-0.5em{\scshape i\kern-0.25em b}\kern-0.8em\TeX}}}

%% Rights management information.  This information is sent to you
%% when you complete the rights form.  These commands have SAMPLE
%% values in them; it is your responsibility as an author to replace
%% the commands and values with those provided to you when you
%% complete the rights form.
\setcopyright{acmlicensed}
\copyrightyear{2024}
\acmYear{2024}
\acmDOI{XXXXXXX.XXXXXXX}

% These commands are for a PROCEEDINGS abstract or paper.
\acmConference[DAC'24]{61st ACM/IEEE Design Automation Conference}{June 23--27, 2024}{San Francisco, CA}

%%
%% Submission ID.
%% Use this when submitting an article to a sponsored event. You'll
%% receive a unique submission ID from the organizers
%% of the event, and this ID should be used as the parameter to this command.
\acmSubmissionID{1122}

\widowpenalties 5 100 80 60 40 20
\raggedbottom

\begin{document}

%%
%% The "title" command has an optional parameter,
%% allowing the author to define a "short title" to be used in page headers.

\title{\acmsmall EDGE-LLM: Enabling Efficient Large Language Model Adaptation on Edge Devices via Layerwise Unified Compression and Adaptive Layer Tuning \& Voting}

% \newcommand{\titleorig}{EDGE-LLM: Enabling Efficient Large Language Model Adaptation on Edge Devices via Layerwise Unified Compression and Adaptive Layer Tuning \& Voting}


\author{\small Zhongzhi Yu$^1$, Zheng Wang$^1$, Yuhan Li$^1$, Xiaoya Zhou$^3$, Haoran You$^1$,\\ Ruijie Gao$^1$, Sreenidhi Reedy Bommu$^1$, Yang (Katie) Zhao$^2$, Yingyan (Celine) Lin$^1$}
\affiliation{%
  \institution{$^1$\textit{Georgia Institute of Technology}, $^2$\textit{University of Minnesota, Twin Cities},
  $^3$\textit{University of California, Santa Barbara}}
  \city{\{zyu401, zwang2478, yli3326, hyou37, eiclab.gatech, sbommu3, celine.lin\}@gatech.edu,\\ yangzhao@umn.edu, xiaoyazhou@umail.ucsb.edu}
  \country{}
}

%%
%% The "author" command and its associated commands are used to define
%% the authors and their affiliations.
%% Of note is the shared affiliation of the first two authors, and the
%% "authornote" and "authornotemark" commands
%% used to denote shared contribution to the research.


%%
%% By default, the full list of authors will be used in the page
%% headers. Often, this list is too long, and will overlap
%% other information printed in the page headers. This command allows
%% the author to define a more concise list
%% of authors' names for this purpose.
\renewcommand{\shortauthors}{Zhongzhi Yu, et al.}
\renewcommand{\shorttitle}{Edge-LLM}

%%
%% The abstract is a short summary of the work to be presented in the
%% article.

% \settopmatter{printacmref=false} % Removes citation information below abstract
% \renewcommand\footnotetextcopyrightpermission[1]{} % removes footnote with conference information in first column
% \pagestyle{plain} % removes running headers
% \vspace{-1em}
\input{Sections/0-Abstract}



%% A "teaser" image appears between the author and affiliation
%% information and the body of the document, and typically spans the
%% page.
% \begin{teaserfigure}
%   \includegraphics[width=\textwidth]{sampleteaser}
%   \caption{Seattle Mariners at Spring Training, 2010.}
%   \Description{Enjoying the baseball game from the third-base
%   seats. Ichiro Suzuki preparing to bat.}
%   \label{fig:teaser}
% \end{teaserfigure}

% \received{20 February 2007}
% \received[revised]{12 March 2009}
% \received[accepted]{5 June 2009}

%%
%% This command processes the author and affiliation and title
%% information and builds the first part of the formatted document.
\maketitle

\input{Sections/1-Introduction}

\input{Sections/2-Preliminary}

\input{Sections/3-Algorithm}

\input{Sections/4-Hardware}

\input{Sections/5-Evaluation}

\input{Sections/6-Conclusion}

%%
%% The next two lines define the bibliography style to be used, and
%% the bibliography file.
\vspace{-1em}
\bibliographystyle{ACM-Reference-Format}
\bibliography{sample-base}

%%

\end{document}
\endinput
%%
%% End of file `sample-sigplan.tex'.
